\documentclass[]{article}
\usepackage[]{apacite}
\usepackage{url} %this package should fix any errors with URLs in refs.
\usepackage{graphicx}
\usepackage{float}
\usepackage{mathtools}
\usepackage{amsmath}
\usepackage{amssymb}
\usepackage[export]{adjustbox}
\usepackage[authoryear]{natbib}
\bibliographystyle{plainnat}
\usepackage[version=4]{mhchem}

%opening
\title{Papers to read/cite for Str2LeadTrans Paper}
\author{}

\begin{document}

\maketitle

%\begin{abstract}

%\end{abstract}

\section{}

\indent	In \textbf{\citep{Luque:2010}}, 
	
	\begin{itemize}
		\item[(a)] They write, ``As discussed in the previous section, the densities of charged particles increase like N so an exponentially increasing number of electrons is liberated around the head. These electrons drift upwards, where the ion density is basically frozen (neglecting small variations due to attachment). \textbf{Hence at some point the electron density surpasses the ion density and creates a net negative charge that is responsible for the increase in the electric field, eventually to values above Ek; then a second ionization wave sets in.}'' The question, I pose is, does a second-ionization wave occur if heating is neglected?
		
		\item[(b)] They also write, ``Other observations in high‐speed sprite imaging also suggest a negative charging of the streamer channel. The first is the emergence of negative (upward‐propagating) streamers, always reported to occur from a previous channel and some milliseconds after the passage of a positive streamer head. In some observations [Stanley et al., 1999; Cummer et al., 2006; Stenbaek‐Nielsen and McHarg, 2008] the emergence of negative streamers coincides with the lower edge of the trailing emissions. ''
	\end{itemize}

	


\noindent	From \textbf{\cite{Liu:2004a}}: 
	
	\begin{itemize}
		\item[(a)] ``These processes, however, are known to be important for the dynamics of long streamers developing in point-to-plane discharge gaps in low electric fields ($<$E$_k$) \cite{Morrow:1997b}.''
	\end{itemize}

\noindent	From \textbf{\cite{Shi:2016a}}:
	
	\begin{itemize}
		\item [(a)] ``On the other hand, the exponential growth of a streamer may be a property particularly important for lightning initiation, since the current flowing in the channel also exponentially increase \cite{Liu:2010a}, potentially accelerating the heating processes in the discharge channel. It is thus necessary to investigate whether the streamer initiated from a hydrometeteor will exponentially grow over a long distance as it was assumed in \citet{Liu:2012c} and \citet{Sadighi:2015a} and to obtain its propagation characteristics.''
	\end{itemize}



\bibliography{ref.bib}
\end{document}
